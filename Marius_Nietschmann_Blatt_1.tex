\documentclass[11pt]{article}
\usepackage[T1]{fontenc}
\usepackage[german]{babel}
\usepackage[utf8]{inputenc}

%% margins
\usepackage{geometry}
\geometry{
  left=2.5cm,
  right=2.5cm,
  top=2.5cm,
  bottom=2.5cm
}

%% ams/thm
\usepackage{amsmath, amssymb, amsthm, mathtools, cancel, graphicx, wrapfig}

%%\begin{thm config/technicalities}
\theoremstyle{plain}
\newtheorem{thm}{Satz}[section]
\newtheorem{lem}[thm]{Lemma}
\newtheorem*{prop}{Proposition}
\newtheorem*{cor}{Korollar}
\newtheorem*{beh}{Behauptung}

\theoremstyle{definition}
\newtheorem{defn}[thm]{Definition}
\newtheorem{exmp}[thm]{Beispiel}

\theoremstyle{remark}
\newtheorem*{rem}{Bemerkung}

\usepackage{thmtools}

\declaretheoremstyle[%
  spaceabove=1.5ex,%
  spacebelow=6pt,%
  headfont=\normalfont\itshape,%
  postheadspace=1em,%
  qed=\qedsymbol%
]{mystyle} 
\declaretheorem[name={Beweis},style=mystyle,unnumbered,
]{prf}

\renewcommand{\proof}{\prf}

%% paragraph spacing/indentation
\usepackage{parskip}
\usepackage{setspace}
\setlength{\parindent}{0cm}

\makeatletter
\def\thm@space@setup{%
  \thm@preskip=\parskip \thm@postskip=0pt
}
\makeatother
%\end{}

%% plotting
%\usepackage{pgfplots}
%\usepackage{tikz}

\usepackage{enumitem}
%\setitemize{noitemsep,topsep=0pt,parsep=0pt,partopsep=0pt}

%% lightning
\usepackage{marvosym}

%% widecheck 
%\usepackage{mathabx}

%% sans-serif font
%\renewcommand{\familydefault}{\sfdefault}

% shortcuts
\newcommand{\N}{\mathbb{N}}
\newcommand{\Z}{\mathbb{Z}}
\newcommand{\Q}{\mathbb{Q}}
\newcommand{\R}{\mathbb{R}}
\newcommand{\C}{\mathbb{C}}
\newcommand{\K}{\mathbb{K}}
\newcommand{\Sn}{\mathbb{S}_n}
\renewcommand{\S}{\mathcal{S}}

\newcommand{\D}{\displaystyle}

\newcommand{\ep}{\varepsilon}
\newcommand{\ph}{\varphi}
\newcommand{\longto}{\longrightarrow}
\DeclareMathOperator{\kgV}{\mathrm{kgV}}
\DeclareMathOperator{\ggT}{\mathrm{ggT}}
\DeclareMathOperator{\ord}{\mathrm{ord}\,}
\DeclareMathOperator{\sgn}{\mathrm{sgn}\,}
\newcommand{\id}{\mathrm{id}}

\DeclareMathOperator{\Fou}{\mathcal{F}\!}

% fancy head/foot
\usepackage{fancyhdr}
\fancyhead[L]{Marius Nietschmann} 
\fancyhead[C]{Blatt 1}
\fancyhead[R]{29. April 2020}


\begin{document}
\pagestyle{fancy}
\thispagestyle{plain}

\rule{\textwidth}{.5pt}
\begin{center}
\Huge{Numerik II - Übungsblatt 1}
\end{center}

\rule{\textwidth}{.5pt}
\text{} \hfill Marius Nietschmann 


%%%%%%%%%%%%%%%%%%%%%%%%%%%%%%%
%%							 %%
%%   !!!Hier geht's los!!!   %%
%%							 %%
%%%%%%%%%%%%%%%%%%%%%%%%%%%%%%%

\section*{Aufgabe 1}

Gegeben sei die Abbildung 
\[ F : \R^2 \to \R^2, (x,y) \longmapsto 
\begin{pmatrix}
	f_1 (x,y) \\ 
	f_2 (x,y) 
\end{pmatrix} =
\begin{pmatrix}
	x^2 + y^2 - 1 \\ 
	x^3 - y + .2
\end{pmatrix}. \] 

\begin{itemize} 
\item[a)] 
Wir stellen zunächst die Kurven $ f_i (x,y) = 0 $, $ i = 1,2 $ grafisch dar. 

\begin{figure}[h]
	\centering 
	\includegraphics[width=7cm]{Übungsblatt 1 Aufgabe 1a Kurven}
\end{figure}

Offenbar hat die Gleichung $ F(x,y) = 0 $ genau 2 Lösungen. Als Näherungswert für eine der Lösungen ist $ a_0 = (.8,.6) $ abzulesen. 
\item[b)] 
Wir wollen nun $ F(x,y) = 0 $ lösen und nutzen hierzu das Newtonverfahren mit Startwert $ a_0 = (.8,.6) $. Im Folgenden runden wir auf 5 Nachkommastellen. Weiter bezeichne $ JF $ die Jacobi-Matrix von $ F $ und $ \Delta a_k = a_k - a_{k-1} $. 
\begin{itemize}
\item[I)] 
Löse $ JF(a_0) \cdot \Delta a_1^T = -F(a_0) $: 
\[ \begin{pmatrix}
	1.60000 & 1.20000 \\ 
	1.92000 & -1.00000
\end{pmatrix} \cdot \Delta a_1^T 
= \begin{pmatrix}
-.00000 \\ 
-.11200
\end{pmatrix}. \] 
Lösung ist $ \Delta a_1 = (-.03443, .04590) $, also $ a_1 = (.76557, .64590) $. 
\item[II)] 
Löse $ JF(a_1) \cdot \Delta a_2^T = -F(a_1) $: 
\[ \begin{pmatrix}
	1.53115 & 1.29180 \\ 
	1.75831 & -1.00000
\end{pmatrix} \cdot \Delta a_2^T 
= \begin{pmatrix}
-.00329 \\ 
-.00280
\end{pmatrix}. \] 
Lösung ist $ \Delta a_2 = (-.00182, -.00039) $, also $ a_2 = (.76376, .64551) $. 
\item[III)] 
Löse $ JF(a_2) \cdot \Delta a_3^T = -F(a_2) $: 
\[ \begin{pmatrix}
	1.52751 & 1.29102 \\ 
	1.74997 & -1.00000
\end{pmatrix} \cdot \Delta a_3^T 
= \begin{pmatrix}
-.00000 \\ 
-.00001
\end{pmatrix}. \] 
Lösung ist $ \Delta a_3 = (-.00000, .00000) $, also $ a_3 = (.76375, .64551) $. 
\end{itemize}
\end{itemize}


\section*{Aufgabe 2}

Gegeben ist nun die Funktion $ P(x) = 3x^3-5x^2-x+1 $. 
\begin{enumerate}
\item[a)] Startwert $ x_0 = 2 $, $ x_{k+1} = x_k - \frac{P(x_k)}{P'(x_k)} $: 
\begin{align*}
x_1 & = 2 - \frac{3}{15} = \frac{9}{5} = 1.8 \\ 
x_2 & = \frac{9}{5} - \frac{\frac{62}{125}}{\frac{254}{25}} = \frac{1112}{635} \approx 1.751 181 102 \\ 
x_3 & \approx 1.751 181 - \frac{.02634}{9.08791} \approx 1.748 282 333 \\ 
x_4 & \approx 1.748 282 - \frac{.00009}{9.02560} \approx 1.748 272 323 \\ 
x_5 & \approx 1.748 272 - \frac{.00000}{9.02538} \approx 1.748 272 323 \\ 
x_6 & \approx 1.748 272 - \frac{.00000}{9.02536} \approx 1.748 272 323 
\end{align*}
\item[b)] Startwert $ x_0 = 0 $: \\ 
$ P(0) = 1, P'(x) = 9x^2 -10x -1 \Longrightarrow P'(0)=-1, P'(0)=-2 $. 
\begin{align*}
x_1 &= x_0 - \frac{P(x_0)}{P'(x_0)} = 0 - \frac{1}{-1} = 1 \\ 
x_2 &= x_1 - \frac{P(x_1)}{P'(x_1)} = 1 - \frac{-2}{-2} = 0 \\ 
x_3 &= 0 - \frac{1}{-1} = 1 \\ 
x_4 &= 1 - \frac{-2}{-2} = 0 \\ 
x_5 &= 0 - \frac{1}{-1} = 1 \\ 
x_6 &= 1 - \frac{-2}{-2} = 0 
\end{align*} 
Offenbar liefert das Newtonverfahren für $ x_0=0 $ alternierend $ 0 $ und $ 1 $. Dies ist daran zu erklären, dass das Newtonverfahren im ersten Schritt $ 0 $ auf $ 1 $ abbildet. Da aber bei $ 1 $ Funktionswert und Ableitung übereinstimmen ($ P(1)=P'(1)=-2 $), also $ \frac{P(1)}{P'(1)}=1 $, und $ 1-1=0 $, liefert das Newtonverfahren, also $ 0 $. Demnach werden $ 0 $ und $ 1 $ immer aufeinander abgebildet und das Newtonverfahren konvergiert nicht. 
\end{enumerate}


\section*{Aufgabe 3}
Gegeben ist folgende Polynomfunktion: 
\[ P(x) = 254 520 x^5 - 580 044 x^4 + 444 746 x^3 - 77 171 x^2 - 77 910 x + 38 955. \] 
\begin{itemize}
\item[a)] Newton ab $ x_0=0 $: 
\begin{align*}
x_0 &= 0 \\ 
x_1 &= 0 - \frac{38 955}{-77 910} = \frac{1}{2} \\ 
x_2 &= \frac{1}{2} - \frac{\frac{16 003}{2}}{-32 006} = \frac{3}{4} \\ 
x_3 &= \frac{3}{4} - \frac{\frac{206115}{128}}{-\frac{618345}{32}} = \frac{5}{6} \\ 
x_4 &= \frac{5}{6} - \frac{\frac{6715}{18}}{-\frac{26860}{3}} = \frac{7}{8} 
\end{align*} 
Man vermutet, das nächste Glied der Newton-Iterationsfolge ist $ \frac{9}{10} $ (und die Folge gegen 1 konvergiert). 
\item[b)] Allerdings: 
\[ x_5 = \frac{7}{8} - \frac{\frac{734069}{4096}}{\frac{104867}{512}} = 0. \] 
Damit wiederholen sich die Folgenglieder periodisch (im Zyklus $ 0, \frac{1}{2}, \frac{3}{4}, \frac{5}{6}, \frac{7}{8} $). 
Insbesondere konvergiert die Iteration nicht. 
\end{itemize}


\section*{Aufgabe 4}

\begin{tabular}{l|c|c|c}
Differentialgeleichung & linear & homogen & konst. Koeff. \\ 
\hline 
a) $ x^2y' = y^2 $ & nein & ja & nein \\ 
b) $ y'''-4y' = 2\sin(\omega x+\ph) $ & ja & nein & ja \\ 
c) $ y^{(4)} = y + x^2\exp(-x) $ & ja & nein & ja \\ 
d) $x^2y'' + xy' + (x^2-1)y = 0 $ & ja & ja & nein \\ 
\end{tabular} 


\section*{Aufgabe 5} 

\begin{itemize}
\item[a)] Löse $ y' = (1-y)^2 $. 
\[ \overset{u = 1-y}{\Longleftrightarrow} u' = \frac{du}{dx} = \frac{du}{dy} \frac{dy}{dx} = -y' = -u^2. \] 
Als Lösung vermuten wir $ u(x) = x^\alpha $ für ein $ \alpha \in \Z $: $ \alpha - 1 = 2\alpha \Longleftrightarrow \alpha = -1 $. \\ 
Also $ u(x) = x^{-1} = \frac{1}{x} $ (Probe: $ u'(x) = \left( \frac{1}{x} \right)' = -\frac{1}{x^2} = -u(x)^2 $). \\ 
Damit ergibt sich für $ y $: 
\[ y(x) = \frac{x-1}{x}. \] 
\item[b)] Löse $ y'(1+x^2) = xy $. 
\[ \overset{1+x^2\neq0}{\Longleftrightarrow} y' = \frac{x}{1+x^2}y. \] 
Bei $ y(c) = a $, $ (c,a) \in\R^2 $,  ergibt sich für die Lösung: 
\begin{align*}
y(x) 
&= a\exp\left(\int_c^x \frac{t}{1+t^2} \, dt \right) 
= a\exp\left(\frac{1}{2} \int_{1+c^2}^{1+x^2} \frac{1}{u} \, du \right) \\ 
&= a\exp\left(\frac{1}{2}\left(\ln\left(1+x^2\right) - \ln\left(1+c^2\right)\right) \right) 
= a\exp\left(\ln\sqrt{\frac{1+x^2}{1+c^2}}\right) \\ 
&= a\sqrt{\frac{1+x^2}{1+c^2}}.
\end{align*}
\item[c)] Löse $ y' + y\cos x = 0 $, $ y\left(\frac{\pi}{2}\right) = 2\pi $. 
\[ y(x) = 2\pi\exp\left(\int_{\frac{\pi}{2}}^x - \cos t \, dt \right) = 2\pi\exp(1-\sin x). \] 
\end{itemize}


\section*{Aufgabe 6}

\begin{itemize}
\item[i)] 
$ \exp(y) \frac{dy}{dx} = \exp(-x)-\exp(y) $, $ y(0)=0 $ 
\item[ii)] 
$ y' = 2\sqrt{|y|} $, $ y(0)=0 $. \\ 
\end{itemize} 

\begin{itemize}
\item[a)] Existenz und Eindeutigkeit von Lösungen der AWP \\ 
Es lässt sich i) ausdrücken als $ y' = \exp(-(x+y))-1 $, $ y(0)=0 $. Die rechte Seite ist in $ y $ stetig differenzierbar, damit erfüllt sie eine lokale Lipschitzbedingung und das AWP besitzt eine eindeutige Lösung (siehe b)). \\ 
Die rechte Seite aus ii) ist stetig, erfüllt aber bei Null keine Lipschitzbedingung, denn es gilt für beliebiges $ \ep > 0 $: 
$ \underset{y\in B_\ep (0)}{\sup} (2\sqrt{|y|})' = \underset{y\in B_\ep (0)}{\sup} \frac{1}{\sqrt{|y|}} = \infty $. Folglich existiert keine eindeutige Lösung der DGL. Offenbar ist nämlich sowohl $ y \equiv 0 $ als auch $ y = x|x| = x^2\sgn x $ Lösung des AWPs. Lösung ist sogar für $ a \leq 0 \leq b \in \R $: 
\[ y(x) = 1_{(b,\infty)}(x) \cdot (x-b)^2 - 1_{(-\infty,a)}(x) \cdot (x-a)^2. \] 
\item[b)] 
Wir lösen nun i). Setze $ u = \exp y $, dann: $ u' = -u + \exp(-x) $, $ u(0)=1 $. \\ 
homogen: $ u_0 (x) = \exp(-x) $, 
inhomogen: $ u(x) = u_0 (x) \left(1+\int\limits_0^x \frac{\exp(-t)}{\exp(-t)} \, dt \right) = (1+x)\exp(-x) $. 
Damit folgt für $ y $: 
\[ y(x) = \ln(x+1)-x. \] 
\end{itemize}

\end{document}